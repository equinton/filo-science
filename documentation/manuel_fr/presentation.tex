\chapter{Le logiciel Filo-Science}
\section{Présentation}

Le logiciel Filo-Science a été conçu pour permettre l'enregistrement des informations concernant les poissons capturés lors des opérations de pêches scientifiques (pêches électriques, pêches au filet). 

Il a été conçu pour l'unité de recherche \textit{Écosystèmes aquatiques et changements globaux} d'IRSTEA, à Cestas (33).

Il a été écrit en PHP, les pages web sont générées en HTML et Javascript avec le composant Smarty.

\section{Fonctionnalités générales}

L'application est organisée autour de la notion de projet : un projet peut regrouper plusieurs campagnes, qui sont menées en réalisant des opérations de pêche. Les droits d'accès sont attribués par projet.

\section{Technologie employée}
\subsection{Base de données}

L'application a été conçue pour fonctionner avec Postgresql, en version 9.6. 

\subsection{Langage de développement et framework utilisé}
Le logiciel a été écrit en PHP, en s'appuyant sur le framework \textit{Prototypephp} \cite{prototypephp}, développé parallèlement par l'auteur du logiciel.

Il utilise la classe \textit{Smarty} \cite{smarty} pour gérer l'affichage des pages HTML. Celles-ci sont générées en utilisant \textit{Jquery} \cite{jquery}  et divers composants associés. Le rendu général est réalisé avec \textit{Bootstrap} \cite{bootstrap}.



\subsection{Liste des composants externes utilisés}
% \usepackage{array} is required
\begin{longtable}{|>{\raggedright\arraybackslash}p{3cm}|c|c|>{\raggedright\arraybackslash}p{3cm}|>{\raggedright\arraybackslash}p{3cm}|}
\hline 
\textbf{Nom} & \textbf{Version} & \textbf{Licence} & \textbf{Usage} & \textbf{Site} \\ 
\hline 
\endhead
PrototypePHP & 5/4/2019 & LGPL & Framework & \href{https://github.com/equinton/prototypephp}{github.com/ equinton/ prototypephp} \\ 
\hline 
Smarty & 3.1 & LGPL & Générateur de pages HTML & \href{http://www.smarty.net}{www.smarty.net} \\ 
\hline 
Smarty-gettext & 1.2.0 & LGPL & Gestion du multi-linguisme avec Smarty & \href{https://github.com/smarty-gettext/smarty-gettext}{https://github.com/smarty-gettext/smarty-gettext} \\
\hline
PHPCAS & 1.3.5 & Apache 2.0 & Identification auprès d'un serveur CAS & \href{https://wiki.jasig.org/display/CASC/phpCAS}{wiki.jasig.org/ display/ CASC/ phpCAS} \\ 
\hline 
Bootstrap & 3.3.7 & MIT & Présentation HTML & \href{http://getbootstrap.com}{get.bootstrap.com} \\ 
\hline 
js-cookie-master & 2.1.4 & MIT & Gestion des cookies dans le navigateur & \href{https://github.com/js-cookie/js-cookie}{github.com/ js-cookie/ js-cookie} \\ 
\hline 
Datatables & 1.10.15 & MIT & Affichage des tableaux HTML & \href{http://www.datatables.net/}{www.datatables. net} \\ 
\hline 
Datetime-moment &  & MIT & Formatage des dates dans les tableaux & \href{https://datatables.net/plug-ins/sorting/datetime-moment}{datatables.net/ plug-ins/ sorting/ datetime-moment} \\ 
\hline 
Moment & 2.8.4 & MIT & Composant utilisé par datetime-moment & \href{http://momentjs.com} {momentjs.com}\\ 
\hline 
JQuery & 3.3.1 & $\approx$ BSD & Commandes Javascript & \href{http://jquery.com/}{jquery.com} \\ 
\hline 
JQuery-ui & 1.12.1 & $\approx$ BSD & Commandes Javascript pour les rendus graphiques & \href{http://jqueryui.com/}{jqueryui.com} \\ 
\hline 
Jquery-timepicker-addon & 1.6.3 & MIT & Time picker & \href{https://github.com/trentrichardson/jQuery-Timepicker-Addon}{github.com/ trentrichardson/ jQuery-Timepicker-Addon} \\ 
\hline 
Magnific-popup & 1.1.0 & MIT & Affichage des photos & \href{http://dimsemenov.com/plugins/magnific-popup/}{dimsemenov .com/plugins/ magnific-popup/}\\ 
\hline 
Smartmenus & 1.1.0 & MIT & Génération du menu HTML & \href{http://www.smartmenus .org}{www.smartmenus .org} \\ 
\hline 
OpenLayers & 4.2.0 & BSD 2 & \href{https://github.com/openlayers/openlayers}{github.com/openlayers/openlayers} \\
\hline
AlpacaJS & 1.5.23 & Apache 2 & Génération et saisie des métadonnées (pour une version future) & 
\href{http://www.alpacajs.org/}{www.alpacajs.org}\\
\hline
\caption{Table des composants externes utilisés dans l'application}
\end{longtable} 

\section{Sécurité}

L'application a été conçue pour résister aux attaques dites opportunistes selon la nomenclature ASVS v4 \cite{asvs} de l'OWASP \cite{owasp}. Des tests d'attaque ont été réalisés en mai 2019 avec le logiciel ZapProxy \cite{zaproxy}, et n'ont pas détecté de faiblesse particulière.

La gestion des droits est conçue pour qu'un utilisateur ne puisse accéder qu'aux projets auxquels il est rattaché. 

\section{Licence}
Le logiciel est diffusé selon les termes de la licence GNU AFFERO GENERAL PUBLIC LICENSE version 3, en date du 19 novembre 2007 \cite{agpl}.

\section{Copyright}

L'application est en cours de dépôt auprès de l'Agence de protection des programmes \cite{app}.
